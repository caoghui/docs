% !TeX spellcheck = en_US
%使用中文:方法1
\documentclass{article}
\usepackage[UTF8]{ctex}
%这里是导言区
%
\title{Git实践}
\author{Bruce}
\date{\today}
\usepackage{graphicx}
%设置页边距
\usepackage{geometry}
%\geometry{papersize={20cm,15cm}}
\geometry{left=1cm,right=2cm,top=3cm,bottom=4cm}
%设置页眉页脚
\usepackage{fancyhdr}
\pagestyle{fancy}
\lhead{\author}
\chead{\today}
\rhead{}
\lfoot{}
\cfoot{\thepage}
\rfoot{}
\renewcommand{\headrulewidth}{0.4pt}
\renewcommand{\headwidth}{\textwidth}
\renewcommand{\footrulewidth}{0pt}

\usepackage[american]{babel}
\usepackage{microtype}
\usepackage{savesym}

\usepackage[bookmarks, colorlinks=true, linkcolor=black, 
    plainpages = false,pdfpagemode = UseNone, pdfstartview = FitH, 
    citecolor = ac, urlcolor = ac, filecolor = ac]{hyperref}

%行间距
\usepackage{setspace}
\onehalfspacing

%段间距
\addtolength{\parskip}{.4em}

%插入源代码 method 1
\usepackage{minted}

%插入源代码 method 2
\usepackage{listings}
\usepackage{xcolor}
\definecolor{dkgreen}{rgb}{0,0.6,0}
\definecolor{gray}{rgb}{0.5,0.5,0.5}
\definecolor{mauve}{rgb}{0.58,0,0.82}
\lstset{ 
    language=c++,
    showspaces=false,
    showtabs=false,
    tabsize=4,
    frame=shadowbox, %single
    framerule=1pt,
    framexleftmargin=5mm,
    framexrightmargin=5mm,
    framextopmargin=5mm,
    framexbottommargin=5mm,
    numbers=left,
    numberstyle=\small\color{gray}, %\small    
    captionpos=b,
    rulecolor=\color{black},
    rulesepcolor= \color{ red!20!green!20!blue!20} ,
	backgroundcolor=\color{white},
    escapeinside=``, % 英文分号中可写入中文
    basicstyle=\footnotesize, %\tt
    directivestyle=\tt,
    identifierstyle=\tt,
    commentstyle=\tt,
    stringstyle=\tt,
    keywordstyle=\color{blue}\tt
}

%设置代码字体
\setmainfont{Times New Roman}
\setsansfont{Arial}
\setmonofont{Courier New}

\newminted{cpp}{gobble=2}

%算法
%\usepackage[ruled]{algorithm2e}

%%%%%正文开始%%%%%%
\begin{document}
\maketitle
\tableofcontents

\newpage


\section{Git配置}
\subsection{设置用户名和邮件地址}

\subsection{设置别名}

\subsection{开启颜色显示}

\subsection{删除配置}
 
\section{常用命令}

\subsection{暂存区}

\subsection{重置}

\subsection{检出}

\section{最佳实践}

\subsection{备份git库}
场景:当已经存在一个git库,为了防止代码库被误删,比较好的做法就是在本地建立一个备份,并经常与代码库同步。
步骤如下:假设已经存在git代码库为[/home/caogh1/workspace/hub],备份库位置:[/media/back]

<1> 使用git init命令创建一个空的裸版本库
\begin{cppcode}
       git init --bare /media/back/hub.git
\end{cppcode}

<2> 使用push命令为其创建内容
\begin{cppcode}
       cd /home/caogh1/workspace/hub
       git push --set-upstream /media/back/hub.git master
\end{cppcode}

<3> 更新工作区内容,使用git push与备份库同步
\begin{cppcode}
       git push
\end{cppcode}

\subsection{处理分离头指针}
场景:当HEAD处于分离头指针时,处理如下。"分离头指针"状态指的就是HEAD头指针指向了一个具体的提交ID,而不是一个引用(分支)。

git checkout master

如果在"分离头指针模式"下有提交,不能通过master分支或其他引用访问。如果提交是master分支所需要的,可以使用合并操作(git merge),具体操作过程如下:

<1> 确认当前处于master分支;

\begin{cppcode}
      git branch -v
\end{cppcode}

<2> 执行合并操作,将提交合并到当前分支;

\begin{cppcode}
     git merge <id>
\end{cppcode}

<3> 仔细看看最新提交,会看到这个提交有两个父提交,这就是合并的奥秘。

\begin{cppcode}
     git cat-file -p HEAD
\end{cppcode}




\end{document}
%%%%%正文结束%%%%%%
